\chapter*{Preface}

Programming is hard. Learning to program is harder. Many books, perhaps out of
commercial pressure, perhaps out of a genuine attempt to make it sound easy,
deny this. You can, indeed, make some simple programs in a few hours. However,
you will not know how to program.

Why should you learn to program? Programming is an intelectually rewarding
activity. Computer programming is pure conceptual play, manipulating logic and
symbols inside your head and trying to work through your intutions and vague
ideas to arrive, finally, at a crisp, exact, expression of them. So, crisp, so
exact, that a machine will give it form.

In its best moments, programming is immensily satisfying (I've heard it being
compared to a mental orgasm, the feeling you get when you finally solve a
puzzle). The only experience that I've sound similar is writing fiction, where
one can also sweat out each word while searching for the best expression of
ideas and emotions.

Realise too that computers, and computer thinking, is at the base of many of
our institutions, something which is only going to be more and more true. To
fully understand today's world, one must understand computers. Not because it
is useful to do so (though it certainly is), but because otherwise, one will
remain alienated from the modern world.

I've often taught of today's world as a world where very few people are
literate. Therefore, a rich pictorical language emerges. We have little symbols
for everything (like most public restrooms are tagged not ``Men'' and
``Women'', but with little symbols). We call these symbols ``making it
user-friendly.'' However, those who learn to read and write (a profoundly
unnatural technology, which takes us years to master) can see and appreciate
the world in a deeper form. Even if you never program professionally, even if
this is the only book you'll ever read on the subject, I hope that by reading
it and learning some of the bases of computers, you will be able to look at the
world a bit differently. Otherwise, there is really no point in reading this
book.

\section*{Notes}

Most of the themes in the Preface, I owe to Peter Norvig whose essay
\emph{Teach Yourself Programming in 10 Years} I agree with and to a blog post
(which I momentarily cannot find) which made some of the same points.

