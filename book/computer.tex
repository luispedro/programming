\newcommand\instruction[1]{\textsc{#1}}%
\chapter{What is a Computer?}

To the post office, it might be a four pound parcel; to the hardware engineer,
a diagram of components, engineered to work together; for a programmer, it is
mostly a composition of three parts: (1) memory, (2) the processor, and (3)
magic.

\section{Magic}

So far, in this chapter, we have referred to many aspects of computers as
\textit{magic}, namely everything that had to deal with input/output (often
abbreviated as \textsc{io}). We now look at how this magic is achieved.

All of the magic is achieved through specialty hardware functionality. For
example, the processor might have a special pair of \instruction{in} and
\instruction{out} instructions, which are used to read and write to devices.
This was used in earlier Intel processors, but recent models tend to use
\emph{memory-mapped \textsc{io}}.

In memory mapped \textsc{io}, the \textsc{cpu} behaves as if it was writing to
a memory address, but instead of actually writing to memory, it sends commands
to the device. Simple devices such as mice or keyboards will have simple
electronics, but others, such as printers, will often have a power processor
and some internal memory of its own. In any case, the device will know how to
handle the incoming commands.

