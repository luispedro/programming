\chapter{What is a Computer?}

In this first chapter, before we actually do any real programming, we will look
more closely at what a computer is.

What is a computer? To the post office, it might be a four pound parcel; to the
hardware engineer, a diagram of components, engineered to work together; for a
programmer, it is mostly a composition of three parts: (1) memory, (2) the
processor, and (3) magic.

Before we even look at actual computers, we will play a simple game, where we
behave like computers.

\section{Assembly Code}%
\newcommand\dorect[1]{\framebox[6em]{\ensuremath{#1}\strut\hfill}}%
\newcommand\dostate[3]{\dorect{#1}\hspace{2pt}\dorect{#2}\hspace{2pt}\dorect{#3}}

Get a sheet of paper and draw three squares (or just follow along here):

\[
\dostate{}{}{}
\]

First, write 3 in the first square, 4 in the second:

\[
\dostate{3}{4}{\strut}
\]

Now perform the following steps:

\begin{enumerate}
\item Write a zero in the third square.
\item Add the number in the second square and the number in the last square
together. Remember it for a second.
\item Erase (or cross-out) the number in the third square and replace it with
the result of your addition.
\item Subtract on from the first square (again, replacing the previous result
by one-minus).
\item If the first square is zero, we are done. Otherwise, go to 2.
\end{enumerate}

\emph{Voilà}, this is your first program. Here is what happens when you perform
the actions. The technical word is \emph{execute}:

\smallskip
\begin{tabular}{ll}
\dostate{3}{4}{0} & (step 1)\\
\dostate{3}{4}{0} & (step 2, 4+0=4)\\
\dostate{3}{4}{\not0 4} & (step 3)\\
\dostate{\not3 \, 2 }{4}{\not0 \, 4} & (step 4)\\
\dostate{\not3 \, 2 }{4}{\not0 \, 4} & (step 5, go to nr.\ 2)\\
\dostate{\not3 \, 2 }{4}{\not0 \, 4} & (step 2, 4+4=8)\\
\dostate{\not3 \, 2 }{4}{\not0 \not4 \, 8} & (step 3)\\
\dostate{\not3 \not2 \, 1}{4}{\not0 \not4 \, 8} & (step 4)\\
\dostate{\not3 \not2 \, 1}{4}{\not0 \not4 \, 8} & (step 5, go to nr.\ 3)\\
\dostate{\not3 \not2 \, 1}{4}{\not0 \not4 \, 8} & (step 2, 4+8=12)\\
\dostate{\not3 \not2 \, 1}{4}{\not0 \not4 \not8 \, 12} & (step 3)\\
\dostate{\not3 \not2 \not1 \, 0}{4}{\not0 \not4 \not8 \, 12} & (step 4)\\
\dostate{\not3 \not2 \not1 \, 0}{4}{\not0 \not4 \not8 \, 12} & (step 5, and we are done)
\end{tabular}
\smallskip

What we are doing here, as you may have figured out, is multiplying the two
numbers in the first cells, by repeated addition ($3\cdot4 = 12$). We
transformed a basic operation (addition), into a slightly more complex
operation (multiplication). This is what programming is: \emph{take some basic
operations, that the computer processor can execute, and build on it to perform
a more complex operation}.\footnote{Nowadays, most processors already have
instructions that perform multiplication, but at one point, they did use
repeated addition.}

Here is how the previous program could look like in a more real programming
language. In fact, this is almost a real programming language, albeit a
severely limited one:

\begin{verbatim}
1. mov M[2], 0
\end{verbatim}

This moves the number zero to the cell $M[2]$. I will call the first cell~0,
the second cell~1, and the third cell~2. This may seem strange, but there are
good reasons for starting to count at zero, so we might as well start
now.\footnote{ E. Dijkstra, a computing pioneer, wrote a long article about
this, available
\href{http://www.cs.utexas.edu/users/EWD/transcriptions/EWD08xx/EWD831.html}{here}.}.

You will also notice that the destination comes first. You can imagine this
being written in the following way:
%
\[
M[2] \leftarrow 0
\]
which might make more sense.

Our program continues:

\begin{verbatim}
2. add M[1], M[2]
\end{verbatim}

The \instruction{add} instruction performs in-place addition (i.e., replaces
the content of its first argument with the result of adding it to its second
argument.\footnote{How could you handle the case where you do not want to lose
any of the arguments, but still get the result of the addition?} In symbolic
notation, it looks like this:

\[
M[1] \leftarrow M[1] + M[2]
\]

The next instruction is a simple one:

\begin{verbatim}
3. dec M[0]
\end{verbatim}

The \instruction{dec} instruction decrements its argument (i.e., subtracts one
from it).

\begin{verbatim}
4. jnz 2
\end{verbatim}

Finally, \instruction{jnz} stands for \emph{jump if not zero}. If the result of
the previous operation (in this case, \instruction{dec}) is \emph{not zero},
then it jumps to the instruction number~2.

\begin{exercise}
How would you perform addition on a machine that has only a \instruction{inc}
and a \instruction{dec} instruction (which increments and decrements its
argument) and the conditional jump instructions \instruction{jz} and
\instruction{jnz}? No \instruction{mov} or \instruction{add} exist in this
hypothetical machine. Your arguments are in cells~0 and~1, and your result
should be in cell~2. It is acceptable to change the values in~$M[0]$ and
$M[1]$.
\begin{solution}
Imagine that the cells contain little glass beads. You take one from the first
cell and put it in the second. Repeat this until there are no beads in the
first cell.

There is a tricky case when either of the arguments are zero. If you have
answered the question before looking at the solution below, make sure that your
solution handles this case correctly.

Here is my proposal:

\begin{verbatim}
1. inc M[0]
2. dec M[0] # This checks whether the argument in M[0] was zero
3. jz 7
4. inc M[2]
5. dec M[0]
6. jnz 4
7. inc M[1]
7. dec M[1]
8. jz 14
9. inc M[2]
10. dec M[1]
10. jnz 9
12.
\end{verbatim}

Note the trick we use at the beginning to check whether an argument is zero:
increment and decrement it. Of course, real processors have an instruction that
can be used to check directly whether a cell is zero or not.
\end{solution}
\end{exercise}

\begin{exercise}
How would you change the program you wrote to make sure that, at the end of it,
the input cells have their original values? (\emph{Hint}: you can use extra
cells as scratch space).
\begin{solution}
The way to approach this is the following: first copy cell 0 into cell 3, and
cell 1 into 4. Then, you can use the previous program to add 3 and~4 into 2.

So, how do you copy a cell? You copy $M[0]$ into two cells ($M[3]$ and $M[4]$)
and then you copy $M[4]$ back. Notice that we temporarily erase the value in
$M[0]$ and then write it back!

Here is how you would copy the cells. We do not check for the case where the
values are zero, but, in a real application, we should (see previous answer for
how to do it):

\begin{verbatim}
1. inc M[3]
2. inc M[4]
2. dec M[0]
3. jnz 1
4. inc M[0]
5. dec M[4]
6. jnz 4
\end{verbatim}
\end{solution}
\end{exercise}

If you understood the above, then you understand the basics of how a computer
works at the lowest level. From now on, we are building up (from these details
up to higher level concepts).

While working through the exercises, you will also get the notion that working
at this level (often called the low-level, where we are manipulating the cells
directly) can get very tedious. Fortunately, in the next chapter we will start
dwelving into real programming languages where the boring details are taken
care of automatically. Right now, we are learning about how the computer works,
actual programming will come later.

\section{Memory \& The Processor}

Normally, we write a number as a sequence of decimal digits. For example, $245$
is interpreted to mean: $2 \times 100 + 4 \times 10 + 5 \times 1$. Starting
from the left, each position has a value that is 10~times larger than the
previous position. Binary numbers, such as $1101_2$ (we will use the little
$_2$-subscript or a $_{10}$ subscript to emphacise which basis we are using) as
the same. The first position (from the left) has value~1, the second position
value~2, then~4, then~8\ldots Another way to put this is that the first
position has value $2^0$, the second $2^1$, the third
$2^2$,\ldots\footnote{Read the mathematical foreword
(page~\pageref{chpt:mathforeword}), if you do not understand the $2^3$
notation.}

\[
1101_2 = 1 \times 2^{0}  + 0 \times 2 + 2^1 \times 4 + 2^2 \times 2^3 = 13_{10}
\]

\begin{exercise}
How would you write the following numbers in binary?

\begin{itemize}
\item 7
\item 8
\item 12
\end{itemize}
\begin{solution}
\begin{itemize}
\item $111_2$
\item $1000_2$
\item $1100_2$
\end{itemize}
\end{solution}
\end{exercise}

\begin{exercise}
How can you quickly check whether a number, written in binary, is odd or even?
\begin{solution}
Even numbers end in a 0, odd numbers in a 1.
\end{solution}
\end{exercise}

Sometimes, too much is made of binary notation. For the moment, all that
matters is that you understand that numbers can be expressed as
binary.

We can now start formalising things a bit more. What we called ``cells'' above
are what is normally called memory. The first thing to realise it that the
cells can only store a limited set of numbers. Physically, each cell is
composed of little wires where you either have electric current or you do not.
If you do, we say that this part of the cell is~1, otherwise, we call
it~0.\footnote{This convention is purely arbitrary and other technologies
exist, some of which have called existence of current~0 and its absence~1. Some
are not based on little wires, but on a magnetic field (North for 1, South for
0; or vice-versa). All of these possibilities, though, are \emph{binary}: one
state for 0, another for 1, and no other possibilities.} Looking at the state
of each wire, we get a \emph{bit} of information. A bit is either a zero or a
one. By convention, each cell has 8~little wires, and can store numbers such as
$01011101_2$ (which is $93_{10}$). If you work it out, the maximum number that
can be stored ($11111111_2$) is 255, for a total of 256 possible values
(including zero).

We call an 8-bit number, a \emph{byte}. There is nothing magical about it being
8~bits. In the past, machines existed with 6~bit bytes or other sizes. But
eight became popular, and is now a widely used convention. Throughout the book,
you will see this pattern again and again: things are a certain way because it
is convenient that we all agree on a common definition, but the definition is
more or less arbitrary.

Thus, each memory cell can save a single byte (a value between 0 and 255). When
we say a computer has a 1~Gigabyte memory, we mean that it has about a trillion
of these cells.

It is also important to note that the cells are ordered. They are not just a
jumbled pile, but more like graph paper. The first cell is numbered~0, the
second cell~1, all the way to 1073741823 (if you have that many in your
computer). This is why we could have programs like the one above, where we
refer to cell 0 (or, in our notation, $M[0]$).

\subsection{About kilos and megas}

If the number 1073741823 in the previous paragraph seems odd, that is because
of definition of Gigabyte.

How many bytes does a kilobyte have? This question used to have a simple,
clear, answer:~1024. It might have been a slight abuse of language to use the
prefix ``kilo'', which normally means 1000 (in the metric system, a kilogram is
1000 grams, a kilometer 1000 metres\ldots), but 1024 was the more meaningful
unit. 1024 is $2^{10}$, which means that, in binary, it is a nice round number
($1000000000_2$), while 1000 is not ($1111101000_2$).

More recently, some felt that this was an unacceptable abuse and that the
metric standard should be followed more closely and this resulted in complete
confusion. While a kilobyte still mostly refers to 1024, sometimes it is now
used to mean 1000 bytes. This is also why a 500~GB drive is identified by your
computer as having less than 500~GB. The manufacturer is just using the
definition that is most convenient for them. At the moment, it is not possible
to always know whether a kilobyte still refers to the old convention (1024) or
the strict metric one (1000).

To try to sort through of this mess, the new words kibibyte (and mebibyte, and
gibibyte) were coined to mean $2^{10}$ (1024), $2^{20}$ and $2^{30}$. I have
actually never heard anyone saying them out loud (so I cannot tell you how to
pronounce them), but many programs use them (mostly their abbreviated forms
KiB, MiB, and GiB). Everyone still says megabyte to mean $2^{20}$ bytes. In
fact, if the difference would matter, we almost always mean the old convention
as decimal notation is foreign to the computer world.

After a few confusing years, we might get a new equilibrium between the old
convention where everybody knew that kilobyte was 1024, the intermediate
confusion where some people insisted on the non-conventional ``kilo is 1000''
and nobody was ever sure what was meant, and a new compromise where we
write ``MiB'' and read it as ``megabyte.''

\section{Processor}

The processor is a device which reads in the program and execute its
instructions. There are two types of instructions: those that change the
contents of memory cells and those that control what parts of the program to
execute next.

In the above, we imagined a processor with some very simple arithmetic
instructions and some very simple conditional jump instructions. Actually,
those are exactly the sort of instruction that a real processor has.

As we have seen, we can use the basic instructions to build more complex ones.
In fact, as we will see, most programming is done using much more complex
languages than the simple one which the processor understands directly. Those
complex constructions (in a more high-level programming language, of which
there are many) are then translated into sequences of very simple instructions
for the program to execute.

The processor takes this sequence of simple instructions and executes them
(that is, it performs the manipulation of values in memory cells that the
program describes).

\section{Representations}

One obvious question is the following: where does this program live, how does
it actually exist? A related question is: how can the computer represent
anything other than numbers? From our previous experience with computers, we
know that computers can save texts, pictures, sounds, and much more. In fact,
it is not even clear how we could represent numbers greater than 255.

To represent numbers greater than 255, we can simply use several cells
together. If we use 2~cells, 2~bytes, we can get numbers with 16~bits (up to 65
thousand, more or less); Using 4~cells, we can get numbers with 32~bits (up to
4 billion, roughly); and with 8~cells, we can get 64~bits. We could, of course,
use a 5 byte number, but, typically, we do not do that, we use multiples of two
(1, 2, 4, 8, \ldots).

So, what about non-numbers?  The answer is very simple: we represent these
things using numbers! If fact, for a computer, nothing else exists but
collections of bytes. For example, for text, we can make up a mapping such that
A is 0, B is 1\ldots and use that to represent a text. Here is how we'd
represent the title of this book:

% title = "PROGRAMMING AN INTRODUCTION"
% print " ".join(map(lambda c: str(ord(c)-ord('A')), title))
15 17 14 6 17 0 12 12 8 13 6 27 0 13 27 8 13 19 17 14 3 20 2 19 8 14 13

In this case, I used 27 to represent a space. Real systems also have a way to
encode punctuation and different values for lower and upper-case letters. We
will spend a whole chapter on this issue as it is conceptually simple, but has
many complex details.

\medskip

Pictures and sounds are also represented as a collection of numbers. A picture
(a digital picture, that is) is a rectangle of pixels (picture elements), which
are small, single-colour blocks. We can represent each pixel with 3 numbers,
one for the amount or red, one for green, and one for blue. Each pixel takes up
3 bytes. If an image has 2048 pixels across and is 1344 pixels tall, then we
need 7.85 MiB to store it. The first byte will be the red component of the
top-left corner, then the green and then the blue components. Repeat a few
million times and you have a representation of an image.\footnote{If the image
files in your computer take up less space, it is because they do not actually
save all of the pixels all the time. Since pixels are often similar to nearby
pixels, you can save some space by not actually storing all of them like this.
We discuss this more in Volume~2.}

We explore representation of data much more in later chapters. We will have a
large discussion just on how to represent text.

What about programs? They are simply numbers too. There is a fixed set of
instructions, so we might have a table assigning numbers to each of them, plus
a way to encode their arguments. For the simple instructions we had before, we
could say that the \instruction{mov} instruction is 0, the \instruction{add}
instruction is 1, and encode its operands similarly.

One question that you might have is the following: if you have a piece of
memory with its values, how do you know whether you should interpret it as text
or as sound or as a program? Is the sequence of numbers 0 13 0 a piece of text
of a computer instruction? You do not know. Memory only hold values,
interpretation must be elsewhere. If you make a mistake, then you get a bunch
of non-sense. You may already have run into an example of this by opening a
file with the wrong program (files are, similarly, just a long sequence of
bytes). The program mis-interprets the information in the file and displays
what looks like garbage.

\medskip

We can now put it all together. A computer has memory and a processor. Part of
the memory encodes the program that the processor is executing. The processor
just performs the following, over and over again: (1) read the memory location
for its next instruction, (2) execute it, eventually changing the value of
other memory locations or the place that it is going to read its next
instruction from.

All of these instructions are very simple, but the processor can execute very
many of them, very fast. A 2.3~GHz processor executes 2.3 trillion instructions
per second. Here is how fast it is: if these words are about 30cm away from
your eyes, then a modern processor will execute two instructions in the time it
takes the light to travel from these pages to your eyes. % This assumes light
% in vaccuum; in air, it would actually be four instructions!

The processor has a few special cells, called \emph{registers} that store the
\emph{instruction pointer}, the memory address of the next instruction and
\emph{flags}. A \emph{flag} is a bit storing special state. For example,
whenever an arithmetic operation results in a zero value, the \emph{zero flag}
is set to true. This is how the \instruction{jnz} operation we saw before
really means: jump if the zero flag is not true. And ``jumping'' is simply a
matter of setting the value of the instruction pointer.

In fact, the registers are often used for one other reason: they are physically
located inside the processor and their use is much faster than accessing the
memory. Therefore, in addition to the special registers outlined above, a
processor will typically have a few \emph{general purpose registers}. The
typical pattern of actually executing an operation is to first load the
relevant values from memory, operate on them in registers, and, finally, write
them back. In fact, some processor architectures only support this mode of
operation (with no operations directly on memory allowed).

Conceptually, though, registers are just a small number of memory cells, some
of which have special meaning (like the instruction pointer), others are just
much faster than the rest of the memory, but otherwise behave just the same.

\section{Programming Languages}

The basic programs we saw above were written in assembly language. Assembly is
just one step above the actual numbers that the computer uses, which is called
\emph{machine language}, it is simply a textual representation of them.

It is not a very flexible programming language and any program you write will
only be adapted to that specific computer architecture.

Therefore, most of the time, programs are written in what are called
\emph{higher-level} languages, where most of the boring work is automatically
done by a program. For example, here is a program in a higher level language

\begin{verbatim}
a = 2
b = 3
print a+4*b
\end{verbatim}

We can write this as a file (a collection of bytes, i.e., numbers such that it
encodes the above text). There is then a program, called an interpreter, which
takes this program and is going to decode it and perform instructions such that
some memory cell will contain the value 2 (which is called \code{a}), another
the value 3 (which is called \code{b}) and then an arithmetic operation is
performed and through \emph{magic}, the result (14) is printed on the screen.

The interpreter takes care of details such as assigning memory locations to
human-readable names and many other details that would be very bothersome to
keep track of and that a computer can do for you. After this chapter, we will
only talk about higher level languages. It is good to know about what happens
under the hood, but often, we will only be interested in working at the higher
level.

Instead of being interpreted, a program can also be \emph{compiled}. This means
that there is another program, the \emph{compiler}, which takes the text of the
program, transforms it into the machine instructions which perform the
operation just described and writes them out in machine code. Then, it can be
directly executed by the machine, which can, potentially, be faster.

The programs you use on a daily basis, such as web browsers or text processing
suites, are a mixture of compiled and interpreted.

At this point, you might be asking: how do you write a compiler or an
interpreter? It seems you would need a compiler for that? How does the first
compiler appear? The general technique is called \emph{bootstrapping} from the
expression \emph{to pull yourself up by your bootstraps}. First, someone writes
a compiler for a very simple language in assembly. Then, in that simple
language, you can write compilers for more complex languages. Then, in that
language, you can write interpreters for other languages. It really is
\emph{turtles all the way down}.

\section{Magic}

The computers we have discussed so far simply manipulate values in memory cells
(either main memory or the special cells called registers). In order to
communicate with the outside world, a computer must be able, however, of
showing different images on a screen, receiving characters from a keyboard,
movement from a mouse, or, even control a robotic arm. These tasks are
collectively known as \textsc{input/output} or \textbf{\textsc{io}}.

All of these tasks, we will collectively refer to using the technical
expression \emph{magic}.

Magic is achieved through specialty hardware functionality. For example, the
processor might have a special pair of \instruction{in} and \instruction{out}
instructions, which are used to read and write to devices. This was used in
earlier Intel processors, but recent models tend to use \emph{memory-mapped
\textsc{io}}.

In memory mapped \textsc{io}, the \textsc{cpu} behaves as if it was writing to
a memory address, but instead of actually writing to memory, it sends commands
to the device. Similarly, reading from memory will actually read input from the
device.

A device like a keyboard will send information as a sequence of bytes
describing the keys pressed, while a mouse might send movements.

\begin{exercise}
Devise a binary encoding for a mouse. Imagine that every 10ms (or so), you must
communicate with the main \textsc{cpu}. What would your communication look like?
\begin{solution}
Here is one possibility: every 10ms, you send how much the mouse moved (in
milimeters) in the $x$ and $y$ directions (up-down and right-left, negative
numbers using two's complement). We can imagine using 16bits, using the first 2
bytes for $x$ and the second 2 bytes for $y$, leading to a 4-byte
communication. For example $(10,0)$ would mean a $10mm$ move up, $(-4,-4)$
would mean a $4mm$ down and $4mm$ left.

Real mice are much more complex, but conceptually it is not that much
different. Nowadays, a mouse will actually use the \textsc{usb} protocol, which
specifies what bytes means and how you communicate in general (whatever the
type of device).
\end{solution}
\end{exercise}

Simple devices such as mice or keyboards will have simple electronics and
simple commands, but others, such as printers, will often have a power
processor and some internal memory of its own. You can send very complex
commands to some printers.

Why do we refer to it as \emph{magic} instead of \emph{sending information to
input/output devices}? The reason is that the programmer is very rarely
interacting with the hardware directly. Most of the time, you will be using
prepared mechanisms to make it easy to do \textsc{io}. For example, all systems
have a method whereby, given a human-readable string (such as a name) in
memory, the will display the string to the user. Of course, underneath, it all
involves drawing characters (more byte manipulation) and then communicating
with a display device. However, that is hidden and we can just assume that
there is this magic functionality that ``displays strings to the user.''

\section{Conclusion}

This chapter was different from all the following chapters. It hopefully
provides the foundation for what follows, where we will start to program.

\textbf{For a programmer, a computer is a collection of memory cells, a
processor which manipulates them, and magic to communicate with the user.}

